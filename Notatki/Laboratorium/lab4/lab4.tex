\documentclass[a4paper,12pt]{article}
\usepackage{polski}
\usepackage[utf8]{inputenc}
\usepackage[OT4]{fontenc}
\usepackage{mathtools}
\usepackage{float}
\usepackage{graphicx}
\usepackage{multirow}
\usepackage{listings}
\usepackage{xcolor}

\newcommand{\h}[1]{\noindent \bf #1 \rm \\ \noindent}
\newcommand{\italic}[1]{\it #1 \rm}

\lstdefinestyle{mystyle}{
	backgroundcolor=\color{lightgray},   
	commentstyle=\color{teal},
	keywordstyle=\color{blue},
	numberstyle=\tiny,
	stringstyle=\color{codepurple},
	basicstyle=\ttfamily\footnotesize,
	breakatwhitespace=false,         
	breaklines=true,                 
	captionpos=b,                    
	keepspaces=true,                 
	numbers=left,                    
	numbersep=5pt,                  
	showspaces=false,                
	showstringspaces=false,
	showtabs=false,                  
	tabsize=2,
}
\lstset{style=mystyle}

\begin{document}

\begin{center}
	\LARGE
	Podstawy Techniki Mikroprocesorowej \\
	\large
	LABORATORIUM 4 
\end{center}
\vspace{1cm}

\h{Instrukcje w sterowniku LCD:}
Mikrokontroler sterownika LCD steruje wyłącznie rejestrami IR (instrukcji) oraz DR (danych). Każda instrukcja sterownika jest przechowywana w tych rejestrach przed wykonaniem. Wykonywanie instrukcji jest kontrolowane poprzez sygnały wysyłane przez mikrokontroler sterownika. Są to sygnały:
\begin{itemize}
	\item \italic{R/$\overline{W}$} - sygnały odczytu/zapisu
	\item \italic{RS} - sygnał wyboru rejestru
	\item \italic{DB0-DB7} - sygnały magistrali danych
\end{itemize}
\vspace{5mm}

\noindent
Większość instrukcji tyczy się głównie operacji na pamięci RAM sterownika. Kolejna instrukcja może być wysłana ze sterownika dopiero w momencie kiedy flaga busy (BF) jest w stanie niskim. Wyjątkiem jest instrukcja sprawdzająca stan flagi busy.\\

\h{Wyczyść ekran:}
Wypełnia wszystkie komórki DDRAM spacjami i ustawia licznik adresowy na pozycję zerową. Ekran zostaje przesunięty do początkowej pozycji o ile został wcześniej przesunięty. Sygnał I/D zostaje ustawiony w stan wysoki (tryb zapisywania). Sygnał S nie jest zmieniany.
\begin{table}[H]
	\centering
	\begin{tabular}{|c|c|c|c|c|c|c|c|c|}
		\hline
		\multicolumn{1}{|c|}{\textbf{RS}} & \multicolumn{1}{c|}{\textbf{R/W}} & \multicolumn{1}{c|}{\textbf{DB7}} & \multicolumn{1}{c|}{\textbf{DB5}} & \multicolumn{1}{c|}{\textbf{DB4}} & \multicolumn{1}{c|}{\textbf{DB3}} & \multicolumn{1}{c|}{\textbf{DB2}} & \multicolumn{1}{c|}{\textbf{DB1}} & \multicolumn{1}{c|}{\textbf{DB0}} \\ \hline
		0                                 & 0                                 & 0                                 & 0                                 & 0                                 & 0                                 & 0                                 & 0                                 & 1                                 \\ \hline
	\end{tabular}
\end{table}

\h{Powrót na początek:}
Ustawia licznik adresowy na pozycję zerową. Zawartość DDRAM nie jest zmieniana.
\begin{table}[H]
	\centering
	\begin{tabular}{|c|c|c|c|c|c|c|c|c|}
		\hline
		\textbf{RS} & \textbf{R/W} & \textbf{DB7} & \textbf{DB5} & \textbf{DB4} & \textbf{DB3} & \textbf{DB2} & \textbf{DB1} & \textbf{DB0} \\ \hline
		0           & 0            & 0            & 0            & 0            & 0            & 0            & 1            & -            \\ \hline
	\end{tabular}
\end{table}

\newpage
\h{Aktywacja trybu kursora:}
Obsługa przejścia kursora w tryb odczytu lub zapisu. Jeżeli I/D jest w stanie wysokim, to kursor jest w trybie zapisu, czyli przemieszcza się w lewo przy wpisaniu znaku na wyświetlacz. Jeżeli I/D jest w stanie niskim, to kursor jest w trybie odczytu, czyli przemieszcza się w prawo przy odczytu znaku z wyświetlacza. Jeżeli sygnał S jest w trybie wysokim, to włączone zostanie przewijanie ekranu.
\begin{table}[H]
	\centering
	\begin{tabular}{|c|c|c|c|c|c|c|c|c|}
		\hline
		\textbf{RS} & \textbf{R/W} & \textbf{DB7} & \textbf{DB5} & \textbf{DB4} & \textbf{DB3} & \textbf{DB2} & \textbf{DB1} & \textbf{DB0} \\ \hline
		0           & 0            & 0            & 0            & 0            & 0            & 1            & I/D          & S            \\ \hline
	\end{tabular}
\end{table}

\h{Włączenie/wyłączenie elementów LCD:}
Jeżeli sygnał D jest w stanie wysokim, to wyświetlacz jest włączony. W przeciwnym wypadku wyłącza się, ale jednocześnie nie jest usuwana zawartość pamięci DDRAM.\\

\noindent
Jeżeli sygnał C jest w stanie wysokim, to kursor jest widoczny. W przeciwnym wypadku nie jest widoczny.\\

\noindent
Jeżeli sygnał B jest w stanie wysokim to kursor miga. W przeciwnym wypadku nie miga.
\begin{table}[H]
	\centering
	\begin{tabular}{|c|c|c|c|c|c|c|c|c|}
		\hline
		\textbf{RS} & \textbf{R/W} & \textbf{DB7} & \textbf{DB5} & \textbf{DB4} & \textbf{DB3} & \textbf{DB2} & \textbf{DB1} & \textbf{DB0} \\ \hline
		0           & 0            & 0            & 0            & 0            & 1            & D            & C            & B            \\ \hline
	\end{tabular}
\end{table}

\h{Konfiguracja funkcji:}
Jeżeli sygnał N jest w stanie wysokim, to wyświetlacz ma dwie linie. W przeciwnym wypadku ma jedną linię.\\

\noindent
Sygnał F służy do ustawienia czcionki. Stan niski, to czcionka 5x8 a wysoki to czcionka 5x10. W trybie 2-liniowym możliwa jest tylko czcionka 5x8.\\

\noindent
Jeżeli sygnał DL jest w stanie wysokim, to sterownik przyjmuje dane 8-bitowe. W przeciwnym wypadku 4-bitowe (DB4-DB7).
\begin{table}[H]
	\centering
	\begin{tabular}{|c|c|c|c|c|c|c|c|c|}
		\hline
		\textbf{RS} & \textbf{R/W} & \textbf{DB7} & \textbf{DB5} & \textbf{DB4} & \textbf{DB3} & \textbf{DB2} & \textbf{DB1} & \textbf{DB0} \\ \hline
		0           & 0            & 0            & 1            & DL           & N            & F            & -            & -            \\ \hline
	\end{tabular}
\end{table}

\newpage
\h{Przesunięcie kursora:}
Pozwala na zmianę pozycji kursora lub ekranu bez potrzeby wczytania lub odczytania znaku. Sygnał R/L steruje kiernkiem poruszania się kursora lub ekranu. Sygnał S/C steruje przełączaniem między trybem przesuwania się kursora lub ekranu. 
\begin{table}[H]
	\centering
	\begin{tabular}{|c|c|c|c|c|c|c|c|c|}
		\hline
		\textbf{RS} & \textbf{R/W} & \textbf{DB7} & \textbf{DB5} & \textbf{DB4} & \textbf{DB3} & \textbf{DB2} & \textbf{DB1} & \textbf{DB0} \\ \hline
		0           & 0            & 0            & 0            & 1            & S/C          & R/L          & -            & -            \\ \hline
	\end{tabular}
\end{table}

\h{Ustawienie adresu CGRAM:}
Ustawia licznik adresowy pamięci CGRAM.
\begin{table}[H]
	\centering
	\begin{tabular}{|c|c|c|c|c|c|c|c|c|c|}
		\hline
		\textbf{RS} & \textbf{R/W} & \textbf{DB7} & \multicolumn{1}{c|}{\textbf{DB6}} & \textbf{DB5} & \textbf{DB4} & \textbf{DB3} & \textbf{DB2} & \textbf{DB1} & \textbf{DB0} \\ \hline
		0           & 0            & 0            & 1                                 & A            & A            & A            & A            & A            & A            \\ \hline
	\end{tabular}
\end{table}

\h{Ustawienie adresu DDRAM:}
Ustawia licznik adresowy pamięci DDRAM. Jeżeli LCD jest w trybie 1-liniowym to adres może wynosić od 00H do 4FH. Jeżeli jest  w 2-liniowym, to pierwszej linii przypisane są adresy od 00H do 27H, a drugiej od 40H do 67H.
\begin{table}[H]
	\centering
	\begin{tabular}{|c|c|c|c|c|c|c|c|c|c|}
		\hline
		\textbf{RS} & \textbf{R/W} & \textbf{DB7} & \textbf{DB6} & \textbf{DB5} & \textbf{DB4} & \textbf{DB3} & \textbf{DB2} & \textbf{DB1} & \textbf{DB0} \\ \hline
		0           & 0            & 1            & A            & A            & A            & A            & A            & A            & A            \\ \hline
	\end{tabular}
\end{table}

\h{Odczytaj flagę busy i adres:}
Pobiera ze sterownika stan flagi busy oraz wartość licznika adresowego w CGRAM i DDRAM.
\begin{table}[H]
	\begin{tabular}{|c|c|c|c|c|c|c|c|c|c|}
		\hline
		\textbf{RS} & \textbf{R/W} & \textbf{DB7} & \textbf{DB6} & \textbf{DB5} & \textbf{DB4} & \textbf{DB3} & \textbf{DB2} & \textbf{DB1} & \textbf{DB0} \\ \hline
		0           & 1            & BF           & A            & A            & A            & A            & A            & A            & A            \\ \hline
	\end{tabular}
\end{table}

\h{Zapis do CG lub DDRAM:}
Zapisuje 8-bitów do pamięci CGRAM lub DDRAM na pozycje wskazywaną przez licznik adresowy. Jeżeli zostało to ustalone, to kursor przy tej operacji przesunie się automatycznie w prawo.
\begin{table}[H]
	\centering
	\begin{tabular}{|c|c|c|c|c|c|c|c|c|c|}
		\hline
		\textbf{RS} & \textbf{R/W} & \textbf{DB7} & \textbf{DB6} & \textbf{DB5} & \textbf{DB4} & \textbf{DB3} & \textbf{DB2} & \textbf{DB1} & \textbf{DB0} \\ \hline
		1           & 0            & D            & D            & D            & D            & D            & D            & D            & D            \\ \hline
	\end{tabular}
\end{table}

\h{Odczyt z CG lub DDRAM:}
Odczytuje 8-bitów z pamięci CGRAM i DDRAM z pozycji wskazywnej przez licznik adresowy. Jeżeli zostało to ustalone, to kursor przy tej operacji przesunie się automatycznie w lewo.
\begin{table}[H]
	\centering
	\begin{tabular}{|c|c|c|c|c|c|c|c|c|c|}
		\hline
		\textbf{RS} & \textbf{R/W} & \textbf{DB7} & \textbf{DB6} & \textbf{DB5} & \textbf{DB4} & \textbf{DB3} & \textbf{DB2} & \textbf{DB1} & \textbf{DB0} \\ \hline
		1           & 1            & D            & D            & D            & D            & D            & D            & D            & D            \\ \hline
	\end{tabular}
\end{table}

\end{document}