\documentclass[a4paper,12pt]{article}
\usepackage{polski}
\usepackage[utf8]{inputenc}
\usepackage[OT4]{fontenc}
\usepackage{mathtools}
\usepackage{float}
\usepackage{graphicx}
\usepackage{multirow}
\usepackage{listings}
\usepackage{xcolor}

\newcommand{\h}[1]{\noindent \bf #1 \rm \\ \noindent}
\newcommand{\italic}[1]{\it #1 \rm}

\lstdefinestyle{mystyle}{
	backgroundcolor=\color{lightgray},   
	commentstyle=\color{teal},
	keywordstyle=\color{blue},
	numberstyle=\tiny,
	stringstyle=\color{codepurple},
	basicstyle=\ttfamily\footnotesize,
	breakatwhitespace=false,         
	breaklines=true,                 
	captionpos=b,                    
	keepspaces=true,                 
	numbers=left,                    
	numbersep=5pt,                  
	showspaces=false,                
	showstringspaces=false,
	showtabs=false,                  
	tabsize=2,
}
\lstset{style=mystyle}

\begin{document}

\begin{center}
	\LARGE
	Podstawy Techniki Mikroprocesorowej \\
	\large
	LABORATORIUM 1 
\end{center}
\vspace{1cm}	

\h{Wartości liczbowe:}
W Asemblerze wartości liczbowe można wyrazić w formacie:
\begin{itemize}
	\item \italic{dziesiętnym} - 1234/1234d
	\item \italic{ósemkowym} - 123o
	\item \italic{szesnastkowym} - 123h, przy czym w wypadku rozpoczynania się od litery (A-F) należy dopisać na początku liczby zero (0)
	\item \italic{dwójkowym} - 0111b
\end{itemize}
\vspace{5mm}

\h{Znaki:}
Znaki są interpretowane jako jedno lub dwubajtowe znaki ASCII. W każdym wypadku należy wpisać interesujący nas znak/znaki w apostrofy.\\

\h{Ciągi znaków:}
Można także deklarować ciągi znaków umieszczając je między apostrofami.\\

\h{Definiowanie stałych:}
Aby zdefiniować stałą należy użyć dyrektywy EQU.
\begin{lstlisting}[language={[x86masm]Assembler}]
<nazwa stalej> EQU <wartosc>
\end{lstlisting}
\vspace{5mm}

\h{Definiowanie zmiennych:}
Aby zdefiniować zmienną należy użyć dyrektywy SET.
\begin{lstlisting}[language={[x86masm]Assembler}]
<nazwa stalej> SET <wartosc>
\end{lstlisting}
\vspace{5mm}

\h{Definiowanie "wskaźników":}
W Asemblerze można deklarować stałe wskazujące na dane obszary w pamięci. Są to:
\begin{itemize}
	\item \italic{BIT} - adres obszaru w IRAM przechowującego bity
	\item \italic{CODE} - adres w pamięci kodowej
	\item \italic{DATA} - adres w pamięci IRAM adresowanej bezpośrednio
	\item \italic{IDATA} - adres w pamięci IRAM adresowanej pośrednio
	\item \italic{XDATA} - adres w pamięci XRAM
\end{itemize}
\vspace{5mm}

\h{Akumulator (Rejestr A):}
Akumulator jest 8-bitowym rejestrem przeznaczonym do krótkoterminowego zapisywania wartości. Do wartości zawartej w rejestrze można dostać się z pomocą jego symbolu \italic{A}. Do adresu akumulatora w pamięci SFR można dostać się poprzez symbol \italic{ACC}.
\begin{lstlisting}[language={[x86masm]Assembler}]
CLR A ; A - symbol akumulatora
PUSH ACC ; ACC - symbol adresu akumulatora
\end{lstlisting}
\vspace{5mm}

\noindent
Można także uzyskać bezpośredni dostęp do adresów poszczególnych bitów akulumatora poprzez operator '.'.
\begin{lstlisting}[language={[x86masm]Assembler}]
SETB ACC.2 ; ustawienie bitu 2 akumulatora w stan wysoki
\end{lstlisting}
\vspace{5mm}

\h{Rejestr B:}
Rejestr B jest 8-bitowym rejestrem pomocniczym. Znajduje on swoje główne zastosowanie w operacjach mnożenia i dzielenia gidze nie może być zastąpiony żadnym innym rejestrem. Nie opłaca się go z tego powodu używać poza tymi przypadkami. Jego symbolem jest litera B. Tak jak akumulator możliwe jest adresowanie jego poszczególnych bitów za pomocą operatora kropki.
\begin{lstlisting}[language={[x86masm]Assembler}]
SETB B.4 ; B - symbol rejestru B
\end{lstlisting}
\vspace{5mm}

\h{Rejestry R0-R7:}
Uniwersalne rejestry znajdujące się domyślnie na pierwszych 8 bajtach wewnętrznej pamięci RAM (można przenieść je jednak do innego baku w pamięci). Każdy z tych rejestrów można przywołać poprzez symbol $Rn$, gdzie $n$ jest numerem rejestru.
\begin{lstlisting}[language={[x86masm]Assembler}]
MOV R2, #1
\end{lstlisting}
\vspace{5mm}

\noindent
Rejestrów $R0$ i $R1$ można użyć w celu przechowywania adresów do innych komórek w pamięci wewnętrznej. \\

\h{Dodawanie:}
Operację dodawania zawsze wykonuje się za pośrednictwem akumulatora. To zawsze w akumulatorze znajduje się pierwszy składnik dodawania i to właśnie tam zapisuje się uzyskana suma. Mamy dwa warianty dodawania. Pierwszym jest dodawanie bez przeniesienia ADD. Drugim dodawanie z przeniesieniem ADDC, które uwzględnia w w dodawaniu przeniesienie z ostatnio wykonanego działania.
\begin{lstlisting}[language={[x86masm]Assembler}]
ADD A, <rejestr> ; dodanie z rejestru
ADD A, <adres z pamieci wewnetrznej> ; dodanie z pamieci
ADD A, @<rejestr 0/1> ; dodanie z adresu zapisanego w rejestrze
ADD A, #<wartosc> ; dodanie stalej

ADDC A, <rejestr> ; A + rejestr + pozyczka z porzedniego dodawania
\end{lstlisting}
\vspace{5mm}

\h{Odejmowanie:}
Operację odejmowania wykonuje się również za pośrednictwem akumulatora. To zawsze w akumulatorze znajduje się odjemna, a następnie obliczona różnica. W wypadku odejmowania zawsze uwzględniana jest pożyczka. Z tego powodu, kiedy chcemy wykonać proste odejmowanie wartości 8-bitowych należy wyzerować flagę pożyczki $C$.
\begin{lstlisting}[language={[x86masm]Assembler}]
SUBB A, <rejestr>
SUBB A, <adres pamieci wewnetrznej>
SUBB A, @<adres zapisany w rejestrach 0/1>
SUBB A, #<wartosc>
\end{lstlisting}
\vspace{5mm}

\h{Mnożenie:}
Operację mnożenia wykonuje się za pośrednictwem akumulatora oraz rejestru B. De facto sprowadza się ono do wymnożenia zawartości tych rejestrów a następnie zapisania 16-bitowego wyniku w formie $B|A$ (w B znajdzie się starszy bajt wyniku, w akumulatorze młodszy). W wypadku wystąpienia przepełnienia aktywuje się flaga CY.
\begin{lstlisting}[language={[x86masm]Assembler}]
MUL AB ; mnozenie
\end{lstlisting}
\vspace{5mm}

\noindent
Ze względu na fakt, że taki sposób mnożenia jest relatywnie wolny to przy mnożeniu przez 2 używa się o wiele wydajniejszego przesunięcia bitowego akumulatora w lewo. Ma ono dwa warianty.
\begin{lstlisting}[language={[x86masm]Assembler}]
RLC A ; A = 2A, w CY 9 bit wyniku
\end{lstlisting}
\vspace{5mm}

\h{Dzielenie:}
Operacje dzielenie wykonuje się za pośrednictwem akumulatora i rejestru B. Dzieli się w nim zawartość akumulatora przez zawartość rejestru. Iloraz zapisuje się w rejestrze A, reszta z dzielenia w rejestrze B. Flaga przeniesienia przy dzieleniu ustawi się na stan wysoki, gdy wynik dzielenie będzie mniejszy od 1.
\begin{lstlisting}[language={[x86masm]Assembler}]
DIV AB ; dzielenie
\end{lstlisting}
\vspace{5mm}

\noindent
Podobnie jak w wypadku mnożenia, polecenie DIV jest dosyć powolne. Dlatego przy dzieleniu przez 2 lepiej jest użyć przesunięcia bitowego akumulatora w prawą stronę.
\begin{lstlisting}[language={[x86masm]Assembler}]
RRC A ; A = A/2, w CY znajduje sie reszta z dzielenia
\end{lstlisting}
\vspace{5mm}

\h{Rozkaz skoku CJNE:}
Rozkaz \it Compare and Jump if Not Equal \rm przeskakuje do wskazanego miejsca w programie w wypadku, gdy zadane mu wartości nie są równe. Może być on użyty do porównania dwóch wartości. Ustawia on flagę CY na stan wysoki, gdy wartość pierwsza jest mniejsza od drugiej. W przeciwnym wypadku flaga CY jest ustawiana na stan niski.
\begin{lstlisting}[language={[x86masm]Assembler}]
CJNE <wartosc 1>, <wartosc 2>, <nazwa etykiety>
; CY = 0 <-> wartosc 1 >= wartosc 2 
; CY = 1 <-> wartosc 1 <  wartosc 2
; Niewykonanie skoku w wypadku rownosci wartosci

CJNE A, <RAM>, <etykieta>
CJNE Rn, #<wartosc>, <etykieta>
CJNE @<R0/1>, #<wartosc>, <etykieta>
\end{lstlisting}
\vspace{5mm}

\h{Dyrektywa CSEG AT:}
Dykretywa informująca kompilator od którego miejsca w pamięci kodowej ma rozpocząć się następna sekcja kodu. Można użyć ją do rozpoczęcia kodu programu, aby wskazać, że jego kod programu rozpoczyna się w komórce zero pamięci kodowej.
\begin{lstlisting}[language={[x86masm]Assembler}]
CSEG AT <adres w pamieci kodowej>
\end{lstlisting}
\vspace{5mm}

\h{Segmenty:}
Segmenty to alokowane przez programistę bloki pamięci. Segmenty dzielimy ze względu na przynależność do pamięci (kodowa, wewnętrzne itd). Możemy je "ręcznie" ustawić w danym miejscu pamięci (segment absolutny) za pomocą dyrektyw:
\begin{itemize}
	\item \italic{BSEG AT} - segment w pamięci bitowej
	\item \italic{CSEG AT} - segment w pamięci kodowej 
	\item \italic{DSEG AT} - segment w pamięci wewnętrznej (adresowanej bezpośrednio)
	\item \italic{ISEG AT} - segment w pamięci wewnętrznej (adresowanej pośrednio)
	\item \italic{XSEG AT} - segment w pamięci zewnętrznej
\end{itemize}
\vspace{5mm}

\noindent
Istnieje jednak także możliwość deklaracji segmentów relokowalnych, których umiejscowieniem w pamięci zajmuje się konsolidator. Aby zlecić utworzenie takiego segmentu należy użyć dyrektyw:
\begin{itemize}
	\item \italic{nazwa SEGMENT BIT} - segment w pamięci bitowej
	\item \italic{nazwa SEGMENT CODE} - segment w pamięci kodowej 
	\item \italic{nazwa SEGMENT DATA} - segment w pamięci wewnętrznej (adresowanej bezpośrednio)
	\item \italic{nazwa SEGMENT IDATA} - segment w pamięci wewnętrznej (adresowanej pośrednio)
	\item \italic{nazwa SEGMENT XDATA} - segment w pamięci zewnętrznej
\end{itemize}
\vspace{5mm}

\noindent
Aby użyć segmentów relokowalnych należy je najpierw wybrać za pomocą dyrektywy RSEG lub CSEG (w przypadku dyrektyw kodowych).\\

\h{Dyrektywa DB/DW:}
Dyrektywy służące do zainicjowania obszaru w pamięci kodowej określonymi wartościami. Możemy zainicjować jeden bajt za pomocą dyrektywy DB lub dwa bajty za pomocą dyrektywy DW.
\begin{lstlisting}[language={[x86masm]Assembler}]
<etykieta>:  DB  <wartosc 1-bajtowa>
<etykieta>:  DW  <wartosc 2-bajtowa>
\end{lstlisting}
\vspace{5mm}

\h{Dyrektywa DBIT/DS:}
Dyrektywy służące do rezerwacji pewnej liczby bajtów pamięci. Dyrektywa DS służy do inicjalizacji w pamięci IRAM miejsca na zmienną o danej nazwie. Można użyć również dyrektywy DBIT, aby zainicjalizować miejsce w pamięci bitowej.
\begin{lstlisting}[language={[x86masm]Assembler}]
<etykieta>: DS <wielkosc w bajtach>
<etykieta>: DBIT <wielkosc w bitach>
\end{lstlisting}
\vspace{5mm}

\h{Dyrektywa NAME:}
Dyrektywa NAME służy do określenie nazwy modułu obiektowego, który powstaje po kompilacji programu.
\begin{lstlisting}[language={[x86masm]Assembler}]
NAME <nazwa>
\end{lstlisting}

\end{document}