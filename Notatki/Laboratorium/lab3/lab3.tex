\documentclass[a4paper,12pt]{article}
\usepackage{polski}
\usepackage[utf8]{inputenc}
\usepackage[OT4]{fontenc}
\usepackage{mathtools}
\usepackage{float}
\usepackage{graphicx}
\usepackage{multirow}
\usepackage{listings}
\usepackage{xcolor}

\newcommand{\h}[1]{\noindent \bf #1 \rm \\ \noindent}
\newcommand{\italic}[1]{\it #1 \rm}

\lstdefinestyle{mystyle}{
	backgroundcolor=\color{lightgray},   
	commentstyle=\color{teal},
	keywordstyle=\color{blue},
	numberstyle=\tiny,
	stringstyle=\color{codepurple},
	basicstyle=\ttfamily\footnotesize,
	breakatwhitespace=false,         
	breaklines=true,                 
	captionpos=b,                    
	keepspaces=true,                 
	numbers=left,                    
	numbersep=5pt,                  
	showspaces=false,                
	showstringspaces=false,
	showtabs=false,                  
	tabsize=2,
}
\lstset{style=mystyle}

\begin{document}

\begin{center}
	\LARGE
	Podstawy Techniki Mikroprocesorowej \\
	\large
	LABORATORIUM 3 
\end{center}
\vspace{1cm}

\h{Rejestr TCON:}
Rejestr Timer Control, pozwalający na konfiguracje liczników. Jest to rejestr adresowany bitowo. Składa się z flag bitowych:
\begin{table}[H]
	\centering
	\begin{tabular}{|c|c|l|}
		\hline
		\textbf{Bit} & \textbf{Symbol} & \multicolumn{1}{c|}{\textbf{Opis}}                                                                                                \\ \hline
		7            & TF1            & \begin{tabular}[c]{@{}l@{}}Flaga przepełnienia Timer 1. Czyszczona\\ po obsłużeniu wyjątku przepełnienia \\ Timer 1.\end{tabular} \\ \hline
		6            & TR1            & \begin{tabular}[c]{@{}l@{}}Bit kontrolny Timer 1. Stan wysoki uruchamia\\ Timer 1. Stan niski zatrzymuje go.\end{tabular}         \\ \hline
		5            & TF0            & \begin{tabular}[c]{@{}l@{}}Flaga przepełniania Timer 0. Czyszczona\\ po obsłużeniu wyjątku przepełnia Timer 0.\end{tabular}       \\ \hline
		4            & TR0            & \begin{tabular}[c]{@{}l@{}}Bit kontrolny Timer 0. Stan wysoki uruchamia\\ Timer 0. Stan niski zatrzymuje go.\end{tabular}         \\ \hline
	\end{tabular}
\end{table}
\vspace{5mm}

\h{Rejestr TMOD:}
Rejestr Timer Mode Control, pozwalający na konfiguracje trybu działania liczników. Nie jest on adresowany bitowo. W celu zmiany poszczególnych bitów należy używać masek i funkcji logicznych.
\begin{table}[H]
	\centering
	\begin{tabular}{|c|l|}
		\hline
		\textbf{Bit} & \multicolumn{1}{c|}{\textbf{Opis}}                                                                           \\ \hline
		7            & Flaga uruchamiająca Timer 1.                                                                                 \\ \hline
		6            & \begin{tabular}[c]{@{}l@{}}Flaga włączająca tryb zliczania pulsów zewnętrznych\\ przez Timer 1.\end{tabular} \\ \hline
		5            & Starszy bit konfigurujący Timer 1.                                                                           \\ \hline
		4            & Młodszy bit konfigurujący Timer 1.                                                                           \\ \hline
		3            & Flaga uruchamiająca Timer 0.                                                                                 \\ \hline
		2            & \begin{tabular}[c]{@{}l@{}}Flaga włączająca tryb zliczania pulsów zewnętrznych\\ przez Timer 0.\end{tabular} \\ \hline
		1            & Starszy bit konfigurujący Timer 0.                                                                           \\ \hline
		0            & Młodszy bit konfigurujący Timer 0.                                                                           \\ \hline
	\end{tabular}
\end{table}
\vspace{5mm}

\newpage
\h{Konfiguracja Timerów za pomocą bitów konfiguracyjnych z TMOD:}
\begin{table}[H]
	\centering
	\begin{tabular}{|c|c|c|l|}
		\hline
		\textbf{M1} & \textbf{M0} & \textbf{Tryb} & \multicolumn{1}{c|}{\textbf{Opis}}                                                                                                  \\ \hline
		0           & 0           & 0             & Tryb 13-bitowy.                                                                                                                     \\ \hline
		0           & 1           & 1             & Tryb 16-bitowy.                                                                                                                     \\ \hline
		1           & 0           & 2             & Tryb 8-bitowy.                                                                                                                      \\ \hline
		1           & 1           & 3             & \begin{tabular}[c]{@{}l@{}}Timer 0 staje się dwoma 8-bitowymi\\ timerami.\\ \\ Timer 1 przestaje generować przerwania.\end{tabular} \\ \hline
	\end{tabular}
\end{table}

\end{document}